\documentclass{article}

\usepackage[utf8]{inputenc}
\usepackage[russian]{babel}
\usepackage[a4paper]{geometry}

\usepackage{indentfirst}
\usepackage[shortlabels]{enumitem}
\usepackage{hyperref}
\usepackage{listings}
\lstset{
  basicstyle=\normalsize\fontencoding{T1}\ttfamily
}


\newcommand{\listingsttfamily}{\fontfamily{Noto Sans Mono}\small}

\setlength\parindent{1.25cm}

\title{Реализация модели магазин-клиент в виде смарт-контракта в среде Ethereum.}
\date{}
\author{Смоляков Павел, группа 21214}

\begin{document}
\maketitle

Для начала распишем возможности, которые должны быть описаны моделью.
\begin{enumerate}[-]
    \item (Для владельца) Создание магазина.
    \item (Для владельца) Добавление товара в магазин.
    \item (Для покупателя) Получение прайс-листа.
    \item (Для покупателя) Покупка товара.
\end{enumerate}


\begin{lstlisting}
      ----money---->      ----money??-->
      <---product---
buyer                shop                owner
      <--pricelist--      <--products---
\end{lstlisting}

В данном отчете продемонстрирована реализация каждой из этих возможностей в смарт-контракте, написанном на языке Solidity.

\section{Инициализация магазина}
Инициализация контракта происходит в пустом конструкторе:
\begin{lstlisting}
constructor() Ownable(msg.sender) {
}    
\end{lstlisting}
После конструкции контракта магазину оказывается назначен владелец, однако нет ни одного товара.

\section{Добавление товара в магазин}
Для добавления товаров владельцем магазина может использоваться следующий метод.
\begin{lstlisting}
function addProduct(
    Product memory _product, 
    ProductInformation memory _info) public onlyOwner {
        ... 
}

struct Product {
    string message;
}

struct ProductInformation {
    uint256 productId;
    string name;
    uint256 priceEther;
}
\end{lstlisting}

\begin{lstlisting}

\end{lstlisting}
    

\end{document}
