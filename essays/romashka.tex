\documentclass{article}

\usepackage[utf8]{inputenc}
\usepackage[russian]{babel}
\usepackage[a4paper]{geometry}

\usepackage{indentfirst}
\usepackage[shortlabels]{enumitem}
% \usepackage{hyperref}

\setlength\parindent{1.25cm}

\title{Анализ политики безопасности предприятия\\ООО <<Ромашка>>}
 
\author{Павел Смоляков, гр. 21214}

\begin{document}
\maketitle

\section{Введение}
Данная работа представляет собой анализ политики безопасности предприятия ООО <<Ромашка>>. Анализ будет производиться по главам, в соответствующем порядке.

\section{Глава ``Политика информационной безопасности''}
В данном разделе описывается общий подход к обеспечению информационном безопасности, используемый в ООО <<Ромашка>>. В целом, на абстрактном уровне упомянуты решения для осуществления защиты от большей части возможных внешних и внутренних угроз. Впрочем, часть пунктов можно подвергнуть критике за недостаточную конкретность: 
\begin{enumerate}
    \item упомянут ``механизм оперативного реагирования'', однако нет никаких уточнений по поводу, собственно, оперативности этого реагирования. Может пройти час, неделя, месяц? А может в зависимости от важности и секретности информации прошедшее время может вариироваться?
    \item ``периодический контроль корректности действий ...'', опять же, не указан период, аналогично предыдущему пункту.
    \item ``контроль корректности ...'', ``контроль целостности ...'' - в данных пунктах никак не указан способ контроля. Автоматический (путем использования специализированного ПО)? Ручной? Смешанный?
\end{enumerate}

Было бы уместным предположить, что данные пункты более подробно раскрываются в следующих разделах документа, однако это актуально лишь для пункта 3.

\section{Глава ``Реализация политики информационной безопасности''}
По данной главе отсутствуют комментарии.

\section{Глава ``Методология и принципы построения политики безопасности''}
В данном разделе описываются, в том числе, принципы построения политики безопасности. Набор принципов весьма всеобъемлющий и их описание исчерпывающе; впрочем, в модели безопасности использовался термин ``уязвимость'', который, на мой взгляд, мог бы быть определен более полно. Помимо предложенных вариантов, это также состояние системы, которое позволяет, как минимум:
\begin{enumerate}
    \item лишить доступа к информации иных пользователей
    \item изменять настройки доступа к информации
    \item использовать ресурсы предприятия для получения доступа к приватной информации, принадлежащей другим предприятиям
\end{enumerate}.

\section{Глава ``Угрозы информационной безопасности ЕСЭДО, методы и средства''}

В данном разделе описана уже более прикладная информация. К сожалению, не обошлось без недоработок. В частности:

\begin{enumerate}
    \item Страница 12, п. 4.2.2., ``организация контроля за работой пользователей''. По тексту ранее упоминались системные администраторы, как отдельная группа лиц; системный администратор как один из источников внутренних угроз также должен быть проконтроллирован.
    \item Страница 13, п. 4.2.3.1., ``Монитор следует располагать таким образом, чтобы исключить возможность просмотра содержимого экрана посторонними лицами...''. Также была бы не лишней возможность оперативного отключения монитора.
    \item Страница 14, п. 4.2.3.1., ``Должен использоваться паролируемый хранитель экрана.''. В идеале, компьютер сам должен переводиться в режим хранителя экрана спустя небольшое время (скажем, не более 2 минут).
    \item Страница 15, п. 4.2.3.5., ``Должны быть рассмотрены следующие рекомендации по защите носителей информации, транспортируемых между территориями...''. Нет пункта, рекомендующего производить верификацию информации, находящейся на носителе, по прибытии в пункт назначения. Например, это может быть механизм ЭЦП.
    \item Страница 16, п. 4.2.3.6., ``защиту системы от наличия и появления нежелательной информации''. Каков механизм реагирования, если нежелательная информация все же появилась?
    \item Страница 17, п. 4.2.3.9., ``Средства управления для борьбы со злонамеренными программными кодами''. Не хватает пункта, к примеру, об отключении от сети заражённого компьютера для предотвращения распространения вируса.
\end{enumerate}

\section{Заключение}
Большая часть документа содержит в себе скорее общие рекомендации, по причине чего иногда не хватает технических подробностей, как в уже приведенных примерах.

\end{document}
